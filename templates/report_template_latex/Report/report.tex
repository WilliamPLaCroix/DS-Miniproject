% This report template is adapted from the IEEE style.
% https://www.ieee.org/conferences/publishing/templates.html

\documentclass[conference]{IEEEtran}
\IEEEoverridecommandlockouts

\usepackage{cite}
\usepackage{amsmath,amssymb,amsfonts}
\usepackage{algorithmic}
\usepackage{graphicx}
\usepackage{textcomp}
\usepackage{xcolor}
\def\BibTeX{{\rm B\kern-.05em{\sc i\kern-.025em b}\kern-.08em
    T\kern-.1667em\lower.7ex\hbox{E}\kern-.125emX}}

\begin{document}

\title{Mini-Project Report Title*}

\author{\IEEEauthorblockN{1\textsuperscript{st} Given Name Surname}
\IEEEauthorblockA{Email Address \\ Matriculation No.}
\and
\IEEEauthorblockN{2\textsuperscript{nd} Given Name Surname}
\IEEEauthorblockA{Email Address \\ Matriculation No.}
\and
\IEEEauthorblockN{3\textsuperscript{rd} Given Name Surname}
\IEEEauthorblockA{Email Address \\ Matriculation No.}
\and
\IEEEauthorblockN{4\textsuperscript{th} Given Name Surname}
\IEEEauthorblockA{Email Address \\ Matriculation No.}
\and
\IEEEauthorblockN{5\textsuperscript{th} Given Name Surname}
\IEEEauthorblockA{Email Address \\ Matriculation No.}
}

\maketitle

\begin{abstract}
With this guide, we give an example of what a write-up for the mini-project report could look like. We provide a \LaTeX template that you 
can use in your own write-up, if you want to. We explain some important points regarding specific types of sections that need to be included in the report.
\end{abstract}


\section{Introduction}
\begin{itemize}
	\item \textbf{Introduce the Problem Statement:} Clearly state the main  problem, question, or objective that your project aims to address.
	\item \textbf{Relevance:} Explain why this topic is important and how it contributes to the existing body of knowledge.
	\item \textbf{Formulate Research Questions:} Emphasize the gap in the current understanding or any issues that your work seeks to explore and/or resolve.
	\item \textbf{Approach Outline:} Briefly describe the research design, methodology, or approach you have used to investigate the problem. 
\end{itemize}
\textbf{Organization} Give the reader an overview of how the paper is organized. Mention the main sections and briefly describe what each section covers. This helps the reader navigate through the paper and understand its logical flow.

\section{Related Work}
\begin{itemize}
	\item \textbf{Identify key themes or subtopics:} Divide the related work into themes or subtopics that are relevant to your research. For each theme, summarize the key studies or research papers that have been conducted in that area. It's important to focus on recent and high-quality sources that directly relate to your research.
	\item \textbf{Compare to Previous Work:} Analyze and evaluate the existing literature by comparing and contrasting different approaches including the current state-of-the-art (SoTA) methodology. Identify any commonalities, discrepancies, or unresolved issues among the studies. This helps to highlight the knowledge gaps or limitations in the existing research.
	\item \textbf{Provide a Summary}: Conclude this section with a summary of the key findings and insights from the previous studies.
\end{itemize}

\section{Procedure}
\begin{itemize}
	\item \textbf{Project Objective:} Restate the main research objective or question that your study aims to address. Make it clear how your research will contribute to filling the identified gap or advancing the current understanding.
	\item \textbf{Research Design:} Describe the overall design of your study. Explain whether it is experimental, observational, qualitative, quantitative, or a combination of methods. Justify your choice of research design and explain how it aligns with your project objectives.
	\item \textbf{Data Analysis:} Explain how you will analyze the collected data to address your research objectives. Describe the specific statistical or analytical techniques you will use, such as regression analysis, content analysis, thematic coding, or any other relevant methods. Justify your choice of analysis methods based on their suitability for your research questions.
\end{itemize}

\section{Experiments}
\begin{itemize}
	\item \textbf{Data Collection:} Describe the process of data collection. Explain how you will gather the necessary data or information. Discuss the sources of data, such as surveys, interviews, experiments, or existing datasets. If applicable, mention any sampling techniques you employed and justify their appropriateness.
	\item \textbf{Explain Features:} Do Exploratory Data Analysis and explain the importance of different features in the dataset. 
	\item \textbf{Performance Metrics:} State the choice of performance metrics and justify why they are appropriate measures of performance for your project.
\end{itemize}

\section{Results}
\begin{itemize}
	\item \textbf{Organize your Results:} Start by structuring your results in a logical and coherent manner. Group related findings together and present them in a way that is easy to understand. You can use subheadings or subsections to organize your results based on different themes or research questions.
	\item \textbf{Descriptive Statistics:} If applicable, provide descriptive statistics to summarize and describe your data. This can include measures such as means, standard deviations, frequencies, or percentages. Presenting descriptive statistics helps to provide a clear overview of your data.
	\item \textbf{Inferential Statistics:} If you conducted statistical tests or analyses to test hypotheses or examine relationships, report the results of these analyses. Include relevant statistical values, such as p-values, confidence intervals, effect sizes, or correlations. Clearly state the outcomes of your statistical tests and their implications for your research objectives.
	\item \textbf{Visual Representations} Utilize tables, charts, graphs, or figures to visually represent your data and findings. These visual representations can help to enhance understanding and make your results more accessible. Ensure that all visual representations are appropriately labeled and clearly presented.
	\item \textbf{Interpretation and Discussion:} Interpret your findings in the context of your research objectives and the existing literature. Explain the significance of your results, highlighting any patterns, trends, or relationships that emerged from your data. Discuss how your findings support or contradict previous research and theories. Address any unexpected or surprising results and provide possible explanations or hypotheses for them.
	\item \textbf{Limitations:} Acknowledge the limitations of your study and the potential impact they may have on the interpretation of your results. Discuss any constraints or factors that may have influenced your findings. Being transparent about limitations helps to maintain the integrity of your work.
\end{itemize}

\section{Conclusion}
\begin{itemize}
	\item \textbf{Recapitulate the Objectives:} Begin by restating the main project objectives or questions that guided your study. This helps to remind the reader of the purpose and scope of your research \cite{sample-article}.
	\item \textbf{Summarize the Key Findings:} Provide a brief summary of the main findings from your study. Highlight the most important results and their implications. However, avoid repeating all the details presented in the Results section. Instead, focus on the key outcomes that directly address your research objectives \cite{sample-web}.
	\item \textbf{Future Research Suggestions:} Identify potential directions for future research based on the limitations or gaps identified in your study. Discuss areas that would benefit from additional investigation or areas where more data or different methodologies could provide deeper insights.
	\item \textbf{Final remarks:} Provide a concise and thoughtful closing statement that summarizes the overall significance of your research. Reflect on the importance of your study and its potential impact on the field. Consider the broader implications of your findings and emphasize the value of your research contribution.
\end{itemize}

\bibliographystyle{IEEEtran}
\bibliography{lit}

\appendix
\begin{itemize}
	\item \textbf{Link to Code}: Provide the link to your code repository. You can use any service to host your code (Github, gdrive etc.). The code should be running, accessible and all of your work must be reproducible. In case your code is not accessible or not in a working state, relevant number of points will be deducted.   
	\item \textbf{Link to Overleaf}: Add a link to the Overleaf project which contains all of the LaTex related files for this report.
	\item \textbf{Additional Figures}: Add any and all additional figures produced during you work with relevant explanations in the appendix.
\end{itemize}

\end{document}
